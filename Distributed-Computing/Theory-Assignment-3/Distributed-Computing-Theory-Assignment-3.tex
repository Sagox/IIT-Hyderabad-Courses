\documentclass[12pt]{article}
\setcounter{secnumdepth}{0}
\usepackage[margin=1in]{geometry} 
\usepackage{amsmath,amsthm,amssymb}
 
\usepackage{listings}
\usepackage{color}

\definecolor{dkgreen}{rgb}{0,0.6,0}
\definecolor{gray}{rgb}{0.5,0.5,0.5}
\definecolor{mauve}{rgb}{0.58,0,0.82}

\lstset{frame=tb,
  aboveskip=3mm,
  belowskip=3mm,
  showstringspaces=false,
  columns=flexible,
  basicstyle={\small\ttfamily},
  numbers=none,
  numberstyle=\tiny\color{gray},
  keywordstyle=\color{blue},
  commentstyle=\color{dkgreen},
  stringstyle=\color{mauve},
  breaklines=true,
  breakatwhitespace=true,
  tabsize=3,
  emph={  
    enterCS,
    releaseCS,
    haveResource
    },
  emphstyle={\color{dkgreen}\bfseries}
}
\begin{document}
\title{CS5320: Distributed Computing\\~\\Theory Assignment 3:\\Solving Consensus using
Byzantine Agreement}
\author{Sagar Jain\\CS17BTECH11034}
\date{}
\maketitle

\subsection*{Solution}
Orignal Assumption: Majority of the processs are non-faulty.\\

On removing this assumption, the following changes are observed:
\begin{itemize}
\item  Termination will still happen since the interactive consistency protocol will eventually terminate. 
\item The entries in all their vectors are the same, so even in this case they will all decide on the same value. \textbf{So, agreement holds when removing the assumption}.
\item If a process $P_i$ is non-faulty, then, since the majority of the processes are non-faulty it can be guaranteed that they will all decide $V_i$ as the $i^{th}$ component of their respective vectors. If we remove the assumption then the problem that occurs is, either by chance or collusion the faulty processes can have a majority for a value which is a different value than what is proposed. \textbf{So the algorithm loses its validity on removing the assumption}.
\end{itemize}
\end{document}