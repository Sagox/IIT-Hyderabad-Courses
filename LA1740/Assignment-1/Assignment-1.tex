% --------------------------------------------------------------
% This is all preamble stuff that you don't have to worry about.
% Head down to where it says "Start here"
% --------------------------------------------------------------
 
\documentclass[12pt]{article}
 
\usepackage[margin=1in]{geometry} 
\usepackage{amsmath,amsthm,amssymb}
 
\begin{document}
\title{Implications of an Ageing Global Population}
\author{Sagar Jain}
\date{}
\maketitle
Population ageing, in simple words, is an increment in the median age of the population of a country. Increasing life expectancy and declining fertility rates while being major achievements themselves are also the primary drivers of population ageing. For better or worse, nearly every country in the world is witnessing a rise in both the average age and percentage of elderly in the population. It is estimated that by the year 2050 there would be more than two billion people over the age of sixty which would be around 22\% of the world population at that point. It would not be an exaggeration to say that on looking back population ageing will be one of the most notable social transformation of this century. The following paragraphs describe the social, economic and political implications of population ageing which make it an issue that cannot be ignored.\\
The first and most obvious impact would be a decline in the working-age population. This could lead to a supply shortage of labour and more so of skilled labour, businesses would find it harder to fill high demand roles. This would have unfavourable impacts on the economy like lower productivity, increased labour expenses, slow business growth, declining innovation caused by declining global competitiveness. Wage inflation could be a direct consequence of the decreased labour supply which would, in turn, increase the prices of commodities eventually turning into a vicious cycle. It is easy to see how we might see higher levels of immigration as a way of tackling this problem.\\
With the retirement age remaining fixed and the average life expectancy increasing there will be a higher number of retired persons as a part of the population, this will increase the dependency ratio. There will be an increased number of people drawing pension benefits and a relatively fewer number of people who work and pay income taxes. The situation would only be worse for countries with a high fiscal deficit. One might fear that it will lead to higher tax rates on the shrinking workforce. This would be discouraging for the working class and also disincentivise investments. Thereby leading to a fall in productivity and growth, and to countries entering secular stagnation, for example, an ageing population is one of the factors for the secular stagnation in Japan.\\
Economies will also have to transition production and investments into goods and services associated with older people. For example, there would be greater demand for retirement homes and health care services. Given that demand for health care rises with age we will see the already high health care spendings as a percentage of the GDP increasing even further. In the case of publicly funded health care systems, it would be imperative that an increase in spending actually improves health care and does not deteriorate other social needs. The healthcare sector will also have to be prepared to handle the higher demands for skilled labour and increased demand for in-home care. Healthcare technologies focussing explicitly on elderly care would also see a steady rise in investments. \\
Finally, with an ageing population we would also see increased inequality. It is easy to see how different the lives of people with a good private-sector pension, and those who rely on a depleting public pension would be. Another determinant of equality would be the situation of the housing sector since homeowners would be much better off than those who continue to rent even after retirement.\\
While I have tried to cover the most important social and economic implications of population ageing, it must be noted that the cultural changes would also be very significant. I personally believe that the impacts of population ageing could be managed effectively with gradual policy changes which could smoothen the many inevitable transitions which we are to go through.


\end{document}