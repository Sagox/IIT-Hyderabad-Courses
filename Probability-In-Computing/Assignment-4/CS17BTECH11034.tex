\documentclass[12pt]{article}
\setcounter{secnumdepth}{0}
\usepackage[margin=1in]{geometry} 
\usepackage{amsmath,amsthm,amssymb}
\usepackage{hyperref}
\usepackage{listings}
\usepackage{color}
\usepackage{advdate}
\definecolor{dkgreen}{rgb}{0,0.6,0}
\definecolor{gray}{rgb}{0.5,0.5,0.5}
\definecolor{mauve}{rgb}{0.58,0,0.82}
\usepackage{graphicx}
\hypersetup{
    colorlinks,
    citecolor=black,
    filecolor=black,
    linkcolor=black,
    urlcolor=blue
}

\lstset{frame=tb,
  aboveskip=3mm,
  belowskip=3mm,
  showstringspaces=false,
  columns=flexible,
  basicstyle={\small\ttfamily},
  numbers=none,
  numberstyle=\tiny\color{gray},
  keywordstyle=\color{blue},
  commentstyle=\color{dkgreen},
  stringstyle=\color{mauve},
  breaklines=true,
  breakatwhitespace=true,
  tabsize=3,
  emph={  
    enterCS,
    releaseCS,
    haveResource
    },
  emphstyle={\color{dkgreen}\bfseries}
}
\begin{document}
\title{CS5120: Probability in Computing\\Assignment 4}
\author{Sagar Jain\\CS17BTECH11034}
\date{}
\maketitle
\section{Solutions}
\begin{enumerate}
\item The probability distribution vector after four time steps is $[0.1938, 0.1772, 0.629 ]$.
\item The expected time from i to j is $j^2 - i^2$.
\item The expected number of steps to reach (1, 1, 1) from (0, 0, 0) is 10.
\item The stationary distribution for a random walk on G is $[2/7, 1/7, 3/14, 1/7, 1/7, 1/14]$.
\item \begin{enumerate}
\item We give a constructive proof to show that there exists a coloring of the graph G with 2 colors such that no triangle is monochromatic. Since there exists a valid 3 colouring,  let the 3 colours used to colour the graph G be, C1, C2 \& C3. Colour all the nodes that are coloured C1 to C2. We claim that there are no monochromatic triangles in the so formed graph.\\
It is easy to see that in the orignal colouring every traingle contains all the 3 colours (any less and it would not be a valid 3 colouring). Now since we have only changed C1 to C2 it follows that every graph contains 2 colours. Hence Proved.
\item \textbf{The Algorithm}
\begin{itemize}
\item Initialise an empty 3-SAT formula.
\item For every 3 vertices, check if they form a triangle, if they do, add these two clauses to the formula $(x_i \lor x_j \lor x_k), \;(\bar{x_i} \lor \bar{x_j} \lor \bar{x_k})$. Here the variable $x_i$ is true if node $i$ is coloured black and is false if node $i$ is coloured white.
\item Solve the 3-SAT using  Schoning's algorithm for 3-SAT
\end{itemize}
First we note that if the 3-SAT formula can be satisfied then it means that for every triangle in the graph both the clauses are satisfied. This would mean that for every triangle we have atleast one white node and one black node, this means that no triangle is monochromatic. A solution always exists since the graph is 3-colourable(follows from previous bit).\\
\textbf{Analysis}\\
The time complexity of step 1 is $O(1)$, the time complexity of step 2 is $O(n^3)$ since there can be at most $C^n_3$ triangles in the graph. As a result of this the number of clauses in the 3-SAT formula are $O(n^3)$. The complexity of step 3 is $O^*(1.33^{n^3})$.\\
Therefore, the overall complexity of the algorithm is $O^*(1.33^{n^3})$. It should be noted that a non probabilistic brute force algoritm would have a complexity of $O(2^n)$.
\end{enumerate}
\end{enumerate}
\end{document}